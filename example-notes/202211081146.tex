\documentclass[example-notes.tex]{subfiles}

\begin{document}
    \note{Теорема Пифагора}
    \addtags{прямоугольный треугольник, Пифагор}

    \begin{definition}
        \label{def:202211081146}
        \emph{Прямоугольным треугольником} называется треугольник, у которого один угол равен 90 градусам.
    \end{definition}

    \begin{theorem}
        \label{thm:202211081146}
        Сумма квадратов катетов равна квадрату гипотенузы, т.е.
        \begin{equation}
            \label{eq:202211081146:piphagor}
            a^2+b^2=c^2.
        \end{equation}
    \end{theorem}
    \begin{proof}
        Доказательство очень простое.
    \end{proof}

    \subsection{Это подраздел}\label{subsec:202211081146:this}

    \Blindtext

    \seealso
    См.~также заметку~\hrefnote{202211081205}.

    Также интересна статья Р.~Мак-Элиса~\cite{j.mceliece1978}.
\end{document}